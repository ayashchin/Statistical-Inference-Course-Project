\documentclass[]{article}
\usepackage{lmodern}
\usepackage{amssymb,amsmath}
\usepackage{ifxetex,ifluatex}
\usepackage{fixltx2e} % provides \textsubscript
\ifnum 0\ifxetex 1\fi\ifluatex 1\fi=0 % if pdftex
  \usepackage[T1]{fontenc}
  \usepackage[utf8]{inputenc}
\else % if luatex or xelatex
  \ifxetex
    \usepackage{mathspec}
  \else
    \usepackage{fontspec}
  \fi
  \defaultfontfeatures{Ligatures=TeX,Scale=MatchLowercase}
\fi
% use upquote if available, for straight quotes in verbatim environments
\IfFileExists{upquote.sty}{\usepackage{upquote}}{}
% use microtype if available
\IfFileExists{microtype.sty}{%
\usepackage{microtype}
\UseMicrotypeSet[protrusion]{basicmath} % disable protrusion for tt fonts
}{}
\usepackage[margin=1in]{geometry}
\usepackage{hyperref}
\hypersetup{unicode=true,
            pdftitle={Statistical Inference Course Project},
            pdfauthor={Ashly Yashchin},
            pdfborder={0 0 0},
            breaklinks=true}
\urlstyle{same}  % don't use monospace font for urls
\usepackage{color}
\usepackage{fancyvrb}
\newcommand{\VerbBar}{|}
\newcommand{\VERB}{\Verb[commandchars=\\\{\}]}
\DefineVerbatimEnvironment{Highlighting}{Verbatim}{commandchars=\\\{\}}
% Add ',fontsize=\small' for more characters per line
\usepackage{framed}
\definecolor{shadecolor}{RGB}{248,248,248}
\newenvironment{Shaded}{\begin{snugshade}}{\end{snugshade}}
\newcommand{\KeywordTok}[1]{\textcolor[rgb]{0.13,0.29,0.53}{\textbf{#1}}}
\newcommand{\DataTypeTok}[1]{\textcolor[rgb]{0.13,0.29,0.53}{#1}}
\newcommand{\DecValTok}[1]{\textcolor[rgb]{0.00,0.00,0.81}{#1}}
\newcommand{\BaseNTok}[1]{\textcolor[rgb]{0.00,0.00,0.81}{#1}}
\newcommand{\FloatTok}[1]{\textcolor[rgb]{0.00,0.00,0.81}{#1}}
\newcommand{\ConstantTok}[1]{\textcolor[rgb]{0.00,0.00,0.00}{#1}}
\newcommand{\CharTok}[1]{\textcolor[rgb]{0.31,0.60,0.02}{#1}}
\newcommand{\SpecialCharTok}[1]{\textcolor[rgb]{0.00,0.00,0.00}{#1}}
\newcommand{\StringTok}[1]{\textcolor[rgb]{0.31,0.60,0.02}{#1}}
\newcommand{\VerbatimStringTok}[1]{\textcolor[rgb]{0.31,0.60,0.02}{#1}}
\newcommand{\SpecialStringTok}[1]{\textcolor[rgb]{0.31,0.60,0.02}{#1}}
\newcommand{\ImportTok}[1]{#1}
\newcommand{\CommentTok}[1]{\textcolor[rgb]{0.56,0.35,0.01}{\textit{#1}}}
\newcommand{\DocumentationTok}[1]{\textcolor[rgb]{0.56,0.35,0.01}{\textbf{\textit{#1}}}}
\newcommand{\AnnotationTok}[1]{\textcolor[rgb]{0.56,0.35,0.01}{\textbf{\textit{#1}}}}
\newcommand{\CommentVarTok}[1]{\textcolor[rgb]{0.56,0.35,0.01}{\textbf{\textit{#1}}}}
\newcommand{\OtherTok}[1]{\textcolor[rgb]{0.56,0.35,0.01}{#1}}
\newcommand{\FunctionTok}[1]{\textcolor[rgb]{0.00,0.00,0.00}{#1}}
\newcommand{\VariableTok}[1]{\textcolor[rgb]{0.00,0.00,0.00}{#1}}
\newcommand{\ControlFlowTok}[1]{\textcolor[rgb]{0.13,0.29,0.53}{\textbf{#1}}}
\newcommand{\OperatorTok}[1]{\textcolor[rgb]{0.81,0.36,0.00}{\textbf{#1}}}
\newcommand{\BuiltInTok}[1]{#1}
\newcommand{\ExtensionTok}[1]{#1}
\newcommand{\PreprocessorTok}[1]{\textcolor[rgb]{0.56,0.35,0.01}{\textit{#1}}}
\newcommand{\AttributeTok}[1]{\textcolor[rgb]{0.77,0.63,0.00}{#1}}
\newcommand{\RegionMarkerTok}[1]{#1}
\newcommand{\InformationTok}[1]{\textcolor[rgb]{0.56,0.35,0.01}{\textbf{\textit{#1}}}}
\newcommand{\WarningTok}[1]{\textcolor[rgb]{0.56,0.35,0.01}{\textbf{\textit{#1}}}}
\newcommand{\AlertTok}[1]{\textcolor[rgb]{0.94,0.16,0.16}{#1}}
\newcommand{\ErrorTok}[1]{\textcolor[rgb]{0.64,0.00,0.00}{\textbf{#1}}}
\newcommand{\NormalTok}[1]{#1}
\usepackage{graphicx,grffile}
\makeatletter
\def\maxwidth{\ifdim\Gin@nat@width>\linewidth\linewidth\else\Gin@nat@width\fi}
\def\maxheight{\ifdim\Gin@nat@height>\textheight\textheight\else\Gin@nat@height\fi}
\makeatother
% Scale images if necessary, so that they will not overflow the page
% margins by default, and it is still possible to overwrite the defaults
% using explicit options in \includegraphics[width, height, ...]{}
\setkeys{Gin}{width=\maxwidth,height=\maxheight,keepaspectratio}
\IfFileExists{parskip.sty}{%
\usepackage{parskip}
}{% else
\setlength{\parindent}{0pt}
\setlength{\parskip}{6pt plus 2pt minus 1pt}
}
\setlength{\emergencystretch}{3em}  % prevent overfull lines
\providecommand{\tightlist}{%
  \setlength{\itemsep}{0pt}\setlength{\parskip}{0pt}}
\setcounter{secnumdepth}{0}
% Redefines (sub)paragraphs to behave more like sections
\ifx\paragraph\undefined\else
\let\oldparagraph\paragraph
\renewcommand{\paragraph}[1]{\oldparagraph{#1}\mbox{}}
\fi
\ifx\subparagraph\undefined\else
\let\oldsubparagraph\subparagraph
\renewcommand{\subparagraph}[1]{\oldsubparagraph{#1}\mbox{}}
\fi

%%% Use protect on footnotes to avoid problems with footnotes in titles
\let\rmarkdownfootnote\footnote%
\def\footnote{\protect\rmarkdownfootnote}

%%% Change title format to be more compact
\usepackage{titling}

% Create subtitle command for use in maketitle
\newcommand{\subtitle}[1]{
  \posttitle{
    \begin{center}\large#1\end{center}
    }
}

\setlength{\droptitle}{-2em}
  \title{Statistical Inference Course Project}
  \pretitle{\vspace{\droptitle}\centering\huge}
  \posttitle{\par}
  \author{Ashly Yashchin}
  \preauthor{\centering\large\emph}
  \postauthor{\par}
  \predate{\centering\large\emph}
  \postdate{\par}
  \date{August 5, 2017}


\begin{document}
\maketitle

\subsection{Part One - Simulation
Excercise}\label{part-one---simulation-excercise}

The following is an investigation into exponential distributions and
their comparison to the Central Limit Theorem (CLT).

For the purpose of this excercise, lambda is assumed to equal 0.2.

Both mean and standard deviation equals 1/lambda.

Let's investigate the distribution of averages of 40 exponentials using
1000 simulations.

Before we begin, let's set a seed for reproducibility. Then let's
determine our theoretical mean of 1/lambda.

\begin{Shaded}
\begin{Highlighting}[]
\KeywordTok{set.seed}\NormalTok{(}\DecValTok{2017}\NormalTok{)}

\CommentTok{# Number of Exponentials}
\NormalTok{n=}\DecValTok{40}

\CommentTok{# Number of Simulations}
\NormalTok{sims=}\DecValTok{1000}

\NormalTok{lambda <-}\StringTok{ }\FloatTok{0.2}
\NormalTok{tmean <-}\StringTok{ }\DecValTok{1}\OperatorTok{/}\NormalTok{lambda}
\NormalTok{tmean}
\end{Highlighting}
\end{Shaded}

\begin{verbatim}
## [1] 5
\end{verbatim}

The answer we get is 5. We divide this by our sample to first get our
standard deviation and then square it to find our variance.

What about our theoretical variance?

\begin{Shaded}
\begin{Highlighting}[]
\NormalTok{tsd <-}\StringTok{ }\NormalTok{tmean}\OperatorTok{/}\KeywordTok{sqrt}\NormalTok{(n)}
\KeywordTok{print}\NormalTok{(}\KeywordTok{paste}\NormalTok{(}\StringTok{"Theoretical Standard Deviation:"}\NormalTok{,}\KeywordTok{round}\NormalTok{(tsd,}\DecValTok{3}\NormalTok{)))}
\end{Highlighting}
\end{Shaded}

\begin{verbatim}
## [1] "Theoretical Standard Deviation: 0.791"
\end{verbatim}

\begin{Shaded}
\begin{Highlighting}[]
\NormalTok{tvar <-}\StringTok{ }\NormalTok{tsd}\OperatorTok{^}\DecValTok{2}
\KeywordTok{print}\NormalTok{(}\KeywordTok{paste}\NormalTok{(}\StringTok{"Theoretical Variance:"}\NormalTok{,tvar))}
\end{Highlighting}
\end{Shaded}

\begin{verbatim}
## [1] "Theoretical Variance: 0.625"
\end{verbatim}

Now, let's calculate our sample mean.

\begin{Shaded}
\begin{Highlighting}[]
\NormalTok{simulations <-}\StringTok{ }\KeywordTok{replicate}\NormalTok{(sims,}\KeywordTok{rexp}\NormalTok{(n,lambda)) }
\end{Highlighting}
\end{Shaded}

This gives us one row for each exponential (40 total) and one column for
each simulation (1000 total). Next, we need to calculate the mean of
each simulation and plot. We will be calculating the mean of each
column, or each simulation

\begin{Shaded}
\begin{Highlighting}[]
\NormalTok{means <-}\StringTok{ }\KeywordTok{apply}\NormalTok{(simulations,}\DecValTok{2}\NormalTok{,mean)}
\KeywordTok{hist}\NormalTok{(means, }\DataTypeTok{breaks=}\DecValTok{25}\NormalTok{, }\DataTypeTok{main =} \StringTok{"Plot of Sample Means vs. Theoretical"}\NormalTok{, }\DataTypeTok{xlab =} \StringTok{"Mean"}\NormalTok{)}
\KeywordTok{abline}\NormalTok{(}\DataTypeTok{v=}\DecValTok{5}\NormalTok{, }\DataTypeTok{col=}\StringTok{"blue"}\NormalTok{, }\DataTypeTok{lwd =} \DecValTok{3}\NormalTok{) }\CommentTok{#blue line represents the theoretical mean}
\KeywordTok{abline}\NormalTok{(}\DataTypeTok{v=}\KeywordTok{mean}\NormalTok{(means),}\DataTypeTok{col=}\StringTok{"red"}\NormalTok{,}\DataTypeTok{lwd=}\DecValTok{3}\NormalTok{) }\CommentTok{#red line represents the sample mean}
\end{Highlighting}
\end{Shaded}

\includegraphics{Statistical_Inference_Course_Project_Report_files/figure-latex/unnamed-chunk-4-1.pdf}
So what is the sample mean?

\begin{Shaded}
\begin{Highlighting}[]
\KeywordTok{mean}\NormalTok{(means)}
\end{Highlighting}
\end{Shaded}

\begin{verbatim}
## [1] 4.982863
\end{verbatim}

The sample mean is 5.00. It is equivalent to our theoretical mean of 5.

This finding supports the Central Limit Theorem in demonstrating that
the mean of a significant sample size (1000) will be equivalent to the
theoretical (population) mean.

Now what about the variance? We remember that our theoretical variance
is 25 (since our standard deviation is 5). Let's calculate our sample
variance.

\begin{Shaded}
\begin{Highlighting}[]
\NormalTok{svar <-}\StringTok{ }\KeywordTok{sd}\NormalTok{(means)}\OperatorTok{^}\DecValTok{2}
\KeywordTok{print}\NormalTok{(}\KeywordTok{paste}\NormalTok{(}\StringTok{"Sample Variance"}\NormalTok{,}\KeywordTok{round}\NormalTok{(svar,}\DecValTok{3}\NormalTok{)))}
\end{Highlighting}
\end{Shaded}

\begin{verbatim}
## [1] "Sample Variance 0.627"
\end{verbatim}

Our sample variance of 0.62 is very close to our theoretical, above, of
0.625.

Since, our calculations show that this excercise supports the Central
Limit Theorem, it follows that the distribution should be normal. To
confirm, we'll draw a quick plot over our histogram from earlier.

\begin{Shaded}
\begin{Highlighting}[]
\KeywordTok{hist}\NormalTok{(means, }\DataTypeTok{prob=}\OtherTok{TRUE}\NormalTok{, }\DataTypeTok{breaks=}\DecValTok{25}\NormalTok{, }\DataTypeTok{main =} \StringTok{"Distribution of Sample Means vs. Theoretical"}\NormalTok{, }\DataTypeTok{xlab =} \StringTok{"Mean"}\NormalTok{) }
\KeywordTok{lines}\NormalTok{(}\KeywordTok{density}\NormalTok{(means),}\DataTypeTok{lwd=}\FloatTok{1.5}\NormalTok{, }\DataTypeTok{col=}\StringTok{"red"}\NormalTok{) }\CommentTok{#red equals sample}
\KeywordTok{curve}\NormalTok{(}\KeywordTok{dnorm}\NormalTok{(x,}\DataTypeTok{mean=}\NormalTok{tmean,}\DataTypeTok{sd=}\NormalTok{tsd),}\DataTypeTok{lwd=}\FloatTok{1.5}\NormalTok{,}\DataTypeTok{col=}\StringTok{"blue"}\NormalTok{,}\DataTypeTok{add=}\OtherTok{TRUE}\NormalTok{) }\CommentTok{#blue equals normal distribution}
\KeywordTok{legend}\NormalTok{(}\StringTok{"topright"}\NormalTok{,}\KeywordTok{c}\NormalTok{(}\StringTok{"Sample"}\NormalTok{,}\StringTok{"Normal Curve"}\NormalTok{),}\DataTypeTok{lty=}\KeywordTok{c}\NormalTok{(}\DecValTok{1}\NormalTok{,}\DecValTok{1}\NormalTok{),}\DataTypeTok{col=}\KeywordTok{c}\NormalTok{(}\StringTok{"red"}\NormalTok{,}\StringTok{"blue"}\NormalTok{))}
\end{Highlighting}
\end{Shaded}

\includegraphics{Statistical_Inference_Course_Project_Report_files/figure-latex/unnamed-chunk-7-1.pdf}
While not identical, our sample means (red) approximate a normal curve
(blue).


\end{document}
